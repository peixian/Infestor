\documentclass[twocolumn]{article}
\usepackage[utf8]{inputenc}
\usepackage[english]{babel}
\usepackage{lipsum}
\usepackage{multicol}
\usepackage{abstract} % Allows abstract customization
\usepackage{footnote}
\usepackage{listings}
\usepackage{url}
\usepackage{dblfnote}
\usepackage{graphicx}
\usepackage[margin=1in]{geometry}
\usepackage{cite}
\usepackage{natbib}
\usepackage{amsmath}
\usepackage{algorithm}
\usepackage[noend]{algpseudocode}
\usepackage[]{units}

\graphicspath{ {../images/} }
\bibliographystyle{acm} 
\setlength{\columnsep}{1cm}

\begin{document}

\twocolumn[\begin{@twocolumnfalse}
  \centerline{\Large\bfseries Breaking Petya - Solving Malware Using a Poor Implementation of Salsa20}
  \vspace{3ex}
  \centerline{Peixian Wang}
  \centerline{May 6, 2016}
  \vspace{3ex}
  
  \begin{abstract}
	  Ransomware has become a relatively profitable development in recent years with the surge in popularity of Bitcoin and other untracable forms of money transactions. In this paper we detail Petya, a recent form of ransomware targeting Windows platforms and NTFS drives. We describe the construction of Petya and the underlying encryption algorithm, Salsa20, and also present one possible solution utilizing Z3, an efficient satisfiabliity modulo theory solver, to defeat Petya. 
  \end{abstract}
   \vspace{3ex}
\end{@twocolumnfalse}]

\section{Introduction}

\bibliography{paper} 

\end{document}